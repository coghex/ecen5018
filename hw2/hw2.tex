\documentclass[12pt, notitlepage, final]{article} 

\newcommand{\name}{Vince Coghlan}

%\usepackage[dvips]{graphics,color}
\usepackage{amsfonts}
\usepackage{amssymb}
\usepackage{amsmath}
\usepackage{latexsym}
\usepackage{enumerate}
\usepackage{amsthm}
%\usepackage{nccmath}
\usepackage{setspace}
\usepackage[pdftex]{graphicx}
\usepackage{epstopdf}
%\usepackage[siunitx]{circuitikz}
\usepackage{tikz}
\usepackage{float}
%\usepackage{cancel} 
\usepackage{setspace}
%\usepackage{overpic}
\usepackage{mathtools}
\usepackage{listings}
\usepackage{color}
%\usepackage{gensymb}

\usetikzlibrary{calc}
\usetikzlibrary{matrix}
\usetikzlibrary{positioning}

\numberwithin{equation}{section}
\DeclareRobustCommand{\beginProtected}[1]{\begin{#1}}
\DeclareRobustCommand{\endProtected}[1]{\end{#1}}
\newcommand{\dbr}[1]{d_{\mbox{#1BR}}}
\newtheorem{lemma}{Lemma}
\newtheorem*{corollary}{Corollary}
\newtheorem{theorem}{Theorem}
\newtheorem{proposition}{Proposition}
\theoremstyle{definition}
\newtheorem{define}{Definition}
\newcommand{\column}[2]{
\left( \begin{array}{ccc}
#1 \\
#2
\end{array} \right)}

\newdimen\digitwidth
\settowidth\digitwidth{0}
\def~{\hspace{\digitwidth}}

\setlength{\parskip}{1pc}
\setlength{\parindent}{0pt}
\setlength{\topmargin}{-3pc}
\setlength{\textheight}{9.0in}
\setlength{\oddsidemargin}{0pc}
\setlength{\evensidemargin}{0pc}
\setlength{\textwidth}{6.5in}
\newcommand{\answer}[1]{\newpage\noindent\framebox{\vbox{{\bf ECEN 5018 Spring 2014} 
\hfill {\bf \name} \vspace{-1cm}
\begin{center}{Homework \#2}\end{center} } }\bigskip }

\DeclareMathOperator*{\argmin}{arg\,min}

%absolute value code
\DeclarePairedDelimiter\abs{\lvert}{\rvert}%
\DeclarePairedDelimiter\norm{\lVert}{\rVert}
\makeatletter
\let\oldabs\abs
\def\abs{\@ifstar{\oldabs}{\oldabs*}}
%
\let\oldnorm\norm
\def\norm{\@ifstar{\oldnorm}{\oldnorm*}}
\makeatother

\def\dbar{{\mathchar'26\mkern-12mu d}}
\def \Frac{\displaystyle\frac}
\def \Sum{\displaystyle\sum}
\def \Int{\displaystyle\int}
\def \Prod{\displaystyle\prod}
%\def \P[x]{\Frac{\partial}{\partial x}}
%\def \D[x]{\Frac{d}{dx}}
\newcommand{\PD}[2]{\frac{\partial#1}{\partial#2}}
\newcommand{\PF}[1]{\frac{\partial}{\partial#1}}
\newcommand{\DD}[2]{\frac{d#1}{d#2}}
\newcommand{\DF}[1]{\frac{d}{d#1}}
\newcommand{\fix}[2]{\left(#1\right)_#2}
\newcommand{\ket}[1]{|#1\rangle}
\newcommand{\bra}[1]{\langle#1|}
\newcommand{\braket}[2]{\langle #1 | #2 \rangle}
\newcommand{\bopk}[3]{\langle #1 | #2 | #3 \rangle}
\newcommand{\Choose}[2]{\displaystyle {#1 \choose #2}}
\newcommand{\proj}[1]{\ket{#1}\bra{#1}}
\def\del{\vec{\nabla}}
\newcommand{\avg}[1]{\langle#1\rangle}
\newcommand{\piecewise}[4]{\left\{\beginProtected{array}{rl}#1&:#2\\#3&:#4\endProtected{array}\right.}
\newcommand{\systeme}[2]{\left\{\beginProtected{array}{rl}#1\\#2\endProtected{array}\right.}
\def \KE{K\!E}
\def\Godel{G$\ddot{\mbox{o}}$del}

\onehalfspacing

\begin{document}

\answer{}

1) \textbf{The Pass Line: }The pass line in craps is one of the most popular
bets in vegas. Craps is a dice game that utilizes two die. Here are the rules:

\begin{itemize}
  \item{The first roll is called the \textit{come out roll}. Two die are rolled and if the sum is 7 or 11 you
      win. If the sum is 2, 3, or 12 you lose. Otherwise, the number you roll \{4, 5, 6, 8, 9, 10\}
      is called your point.}
  \item{If the outcome of the game is not determined on the first roll then the game continues
      until either one of two events happen:}
      \begin{itemize}
        \item{If I roll my point before I roll a 7 I win.}
        \item{If I roll a 7 before I roll my point I lose.}
      \end{itemize}
\end{itemize}

There is no limit on the number of rolls in any given game.

The pass line in craps is an even money bet, meaning that if I bet \$5 and win, I win \$5. If I
lose, I lose \$5. What is the probability of winning given that I bet on the pass line? If I bet
\$5 on the pass line, how much should I expect to have at the end of the game?\\

There are 36 possible rolls.  The probability of rolling a 7 or 11 on your come out roll is
going to equal the probability of rolling a 7 ($\frac{6}{36}$) and the probability of rolling
an 11 ($\frac{2}{36}$).  This means the probability of winning on the come out roll is
$\frac{8}{36}$.  After that we need to go through each number and find the probability of winning
if that number is the point.  Ill start with 4.  The probability of rolling a 4 as your point is
$\frac{3}{36}$.  The probability of rolling a 4 again before rolling a 7 can be seen below:
\[
  \sum_{n=0}^\infty \frac{3}{36}(\frac{27}{36})^n = \frac{1}{3}
\]
This is because you either get your 4, or you get something other than a 4 or a 7. The probability
of winning with this point value is $\frac{1}{3}\frac{3}{36} = \frac{1}{36}$. Since 10 has the same
probabilities of being rolled as 4, it has the same total probability of winning too. We can find
similar values for the other possible points:
\[
  P(4,10) = \frac{1}{36}\text{, } P(5,9) = \frac{2}{45}\text{, } P(6,8) = \frac{25}{396}
\]
Thus the probability of winning on the pass line is:
\[
  P(\text{win}) = \frac{6}{36} + \frac{2}{36} + 2\cdot(\frac{1}{36}) + 2\cdot(\frac{2}{45}) + 2\cdot(\frac{25}{396}) \approx 0.493
\]
These are pretty good odds for vegas, The expected return would be:
\[
  0.493\cdot10 + (1-0.493)\cdot0= \text{\$}4.93
\]

2) \textbf{Consider Two Events: }$A$ and $B$, with Pr($A$) $>0$ and Pr($B$) $>0$. Is the following
sentance TRUE or FALSE or DEPENDENT ON $A$ and $B$:
\[
  \text{If $A$ and $B$ are disjoint, then $A$ and $B$ must be independent.}
\]

Since $A$ and $B$ are disjoint, P($A \cap B$) = 0, since they can never both happen at the same time.
If they were independent then P($A \cap B$) = P($A$)P($B$).  This is not possible unless P($A$) or
P($B$) is 0, which is against the presumption in the question prompt.  The answer is FALSE.

3) \textbf{The Random Variable: }$X$ takes on values \{0,1,2,3\} with probabilities \{0.4,0.2,0.1,?\},
respectively.  Compute the following:

\begin{enumerate}[(a)]
  \item{Pr($X=3$)}\\
    P($X=3$)$\;= 1 - 0.4 - 0.2 - 0.1 = 0.3$
  \item{Pr($X$ is odd)}\\
    P($X$ is odd)$\;=0.2 + 0.3 = 0.5$
  \item{E[$X$]}\\
    E[$X$]$\;=0\cdot0.4+1\cdot0.2+2\cdot0.1+3\cdot0.3 = 1.3$
  \item{E[$1/(X+1)$]}\\
    E[$1/(X+1)$]$\;=0.4/1 + 0.2/2 + 0.1/3 + 0.3/4 = \frac{73}{120}$
\end{enumerate}

\newpage

4) \textbf{The Probabilities: }of the outcome of two coin tosses are:
\begin{center}
\begin{tabular}{ c | c }
  outcome & probability\\
  \hline
  HH & 2/9\\
  HT & 1/9\\
  TH & 4/9\\
  TT & 2/9\\
\end{tabular}
\end{center}

Compute teh following:

\begin{enumerate}[(a)]
  \item{Pr(first toss is $H$)}
    Pr(first toss is $H$)$\;=3/9$
  \item{Pr(second toss is $T$)}
    Pr(second toss si $T$)$\;=3/9$
  \item{Pr(first toss is $H$ $\mid$ second toss is $T$)}
    Pr(first toss is $H$ $\mid$ second toss is $T$)$\;=5/9$
  \item{Are the two coin tosses independent?}
    No, since the probability that the second toss is a $T$ is entirely dependent on what
    the first toss was.  P($HT$) $\neq$ P($TT$).
\end{enumerate}

5) \textbf{The Table Below: }can be interpreted as tossing a pair of 3-sided dice, labeled $X$ and $Y$.

\begin{center}
\begin{tabular}{ r r|c|c|c| }
\cline{3-5}
& 3 & 0 & 0 & 2/8\\
\cline{3-5}
Y & 2 & 0 & 2/8 & 1/8\\
\cline{3-5}
& 1 & 2/8 & 1/8 & 0\\
\cline{3-5}
 & \multicolumn{1}{r}{}
 & \multicolumn{1}{c}{1}
 & \multicolumn{1}{c}{2}
 & \multicolumn{1}{c}{3} \\

 & \multicolumn{1}{r}{}
 & \multicolumn{1}{c}{}
 & \multicolumn{1}{c}{X}
 & \multicolumn{1}{c}{}\\

\end{tabular}
\end{center}

\begin{enumerate}[(a)]
  \item{Pr($X = 2$)$\;=3/8$}
  \item{Pr($Y = 2$)$\;=3/8$}
  \item{Pr($Y = 3 \mid X = 3$)}\\
    Pr($Y = 3 \mid X = 3$)$\;=2/8$
  \item{E[max($X$,$Y$)]}\\
    E[max($X$,$Y$)]$\;=3\cdot2/8+3\cdot1/8+2\cdot{2/8}+2\cdot{1/8}+1\cdot{2/8} = 17/8$
  \item{Are the two tosses independent?}\\
    No they are not, since $X$ cannot be 1 unless $Y$ is 1.

\end{enumerate}

6) \textbf{For Each of the Following 2 $\times$ 2 Games: }the column player is using a randomized
strategy of $L$ with probability $p$ and $R$ with probability $1-p$.  The row player seeks to
optimize the expected payoff.
\begin{itemize}
  \item{BoS:}
    \begin{center}
      \begin{tabular}{r |c|c|}
        \multicolumn{1}{r}{}
        & \multicolumn{1}{c}{B}
        & \multicolumn{1}{c}{S}\\
        \cline{2-3}
        B & 2,1 & 0,0\\
        \cline{2-3}
        S & 0,0 & 1,2\\
        \cline{2-3}
      \end{tabular}
    \end{center}
  \item{Stag hunt:}
    \begin{center}
      \begin{tabular}{r |c|c|}
        \multicolumn{1}{r}{}
        & \multicolumn{1}{c}{Stag}
        & \multicolumn{1}{c}{Hare}\\
        \cline{2-3}
        Stag & 2,2 & 0,1\\
        \cline{2-3}
        Hare & 1,0 & 1,1\\
        \cline{2-3}
      \end{tabular}
    \end{center}
  \item{Typewriter:}
    \begin{center}
      \begin{tabular}{r |c|c|}
        \multicolumn{1}{r}{}
        & \multicolumn{1}{c}{Alt}
        & \multicolumn{1}{c}{Std}\\
        \cline{2-3}
        Alt & 3,3 & 0,0\\
        \cline{2-3}
        Std & 0,0 & 1,1\\
        \cline{2-3}
      \end{tabular}
    \end{center}

\end{itemize}

\begin{enumerate}[(a)]
  \item{Determine the best response of the row player as a function of $p$.}
  \item{For what value of p is the row player indifferent between its "top" action versus its "bottom" action?}
\end{enumerate}
Lets start with the BoS game, the best response of the row player is going to be contingent on the
probability $p$ that the column player is going to play $B$. This is going to be the maximum of:
\[
  p(q\cdot 2 + (1-q)\cdot 0) + (1-p)(q\cdot 0 + (1-q)\cdot 1)
\]
\[
  2q < (1-q) \text{ when } q < 1/3
\]
\[
  2q > (1-q) \text{ when } q > 1/3
\]

\[
  B_{ROW}(q) = \left\{
     \begin{array}{lr}
       1 & q > 1/3\\
       0 & q < 1/3\\
       \text{[0,1]} & q = 1/3
     \end{array}
   \right.
\]
$1/3$ is the value at which the row player is indifferent.

Now we can look at the Stag hunt game:
\[
  p(q\cdot 2 + (1-q)\cdot 0) + (1-p)(q\cdot 1 + (1-q)\cdot 1)
\]
\[
  2q < 1 \text{ when } q < 1/2
\]
\[
  2q > 1 \text{ when } q > 1/2
\]
\[
  B_{ROW}(q) = \left\{
     \begin{array}{lr}
       1 & q > 1/2\\
       0 & q < 1/2\\
       \text{[0,1]} & q = 1/2
     \end{array}
   \right.
\]
$1/2$ is the value at which the row player is indifferent.

Now we can look at the Typewriter game:
\[
  p(q\cdot 3 + (1-q)\cdot 0) + (1-p)(q\cdot 0 + (1-q)\cdot 1)
\]
\[
  3q < (1-q) \text{ when } q < 1/4
\]
\[
  3q > (1-q) \text{ when } q > 1/4
\]
\[
  B_{ROW}(q) = \left\{
     \begin{array}{lr}
       1 & q > 1/4\\
       0 & q < 1/4\\
       \text{[0,1]} & q = 1/4
     \end{array}
   \right.
\]
$1/4$ is the value at which the row player is indifferent.



\end{document}
