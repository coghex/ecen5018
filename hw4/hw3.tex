\documentclass[12pt, notitlepage, final]{article} 

\newcommand{\name}{Vince Coghlan}

%\usepackage[dvips]{graphics,color}
\usepackage{amsfonts}
\usepackage{amssymb}
\usepackage{amsmath}
\usepackage{latexsym}
\usepackage{enumerate}
\usepackage{amsthm}
%\usepackage{nccmath}
\usepackage{setspace}
\usepackage[pdftex]{graphicx}
\usepackage{epstopdf}
%\usepackage[siunitx]{circuitikz}
\usepackage{tikz}
\usepackage{float}
%\usepackage{cancel} 
\usepackage{setspace}
%\usepackage{overpic}
\usepackage{mathtools}
\usepackage{listings}
\usepackage{color}
\usepackage{qtree}
%\usepackage{gensymb}

\usetikzlibrary{calc}
\usetikzlibrary{matrix}
\usetikzlibrary{positioning}

\numberwithin{equation}{section}
\DeclareRobustCommand{\beginProtected}[1]{\begin{#1}}
\DeclareRobustCommand{\endProtected}[1]{\end{#1}}
\newcommand{\dbr}[1]{d_{\mbox{#1BR}}}
\newtheorem{lemma}{Lemma}
\newtheorem*{corollary}{Corollary}
\newtheorem{theorem}{Theorem}
\newtheorem{proposition}{Proposition}
\theoremstyle{definition}
\newtheorem{define}{Definition}
\newcommand{\column}[2]{
\left( \begin{array}{ccc}
#1 \\
#2
\end{array} \right)}

\newdimen\digitwidth
\settowidth\digitwidth{0}
\def~{\hspace{\digitwidth}}

\setlength{\parskip}{1pc}
\setlength{\parindent}{0pt}
\setlength{\topmargin}{-3pc}
\setlength{\textheight}{9.0in}
\setlength{\oddsidemargin}{0pc}
\setlength{\evensidemargin}{0pc}
\setlength{\textwidth}{6.5in}
\newcommand{\answer}[1]{\newpage\noindent\framebox{\vbox{{\bf ECEN 5018 Spring 2014} 
\hfill {\bf \name} \vspace{-1cm}
\begin{center}{Homework \#3}\end{center} } }\bigskip }

\DeclareMathOperator*{\argmin}{arg\,min}

%absolute value code
\DeclarePairedDelimiter\abs{\lvert}{\rvert}%
\DeclarePairedDelimiter\norm{\lVert}{\rVert}
\makeatletter
\let\oldabs\abs
\def\abs{\@ifstar{\oldabs}{\oldabs*}}
%
\let\oldnorm\norm
\def\norm{\@ifstar{\oldnorm}{\oldnorm*}}
\makeatother

\def\dbar{{\mathchar'26\mkern-12mu d}}
\def \Frac{\displaystyle\frac}
\def \Sum{\displaystyle\sum}
\def \Int{\displaystyle\int}
\def \Prod{\displaystyle\prod}
%\def \P[x]{\Frac{\partial}{\partial x}}
%\def \D[x]{\Frac{d}{dx}}
\newcommand{\PD}[2]{\frac{\partial#1}{\partial#2}}
\newcommand{\PF}[1]{\frac{\partial}{\partial#1}}
\newcommand{\DD}[2]{\frac{d#1}{d#2}}
\newcommand{\DF}[1]{\frac{d}{d#1}}
\newcommand{\fix}[2]{\left(#1\right)_#2}
\newcommand{\ket}[1]{|#1\rangle}
\newcommand{\bra}[1]{\langle#1|}
\newcommand{\braket}[2]{\langle #1 | #2 \rangle}
\newcommand{\bopk}[3]{\langle #1 | #2 | #3 \rangle}
\newcommand{\Choose}[2]{\displaystyle {#1 \choose #2}}
\newcommand{\proj}[1]{\ket{#1}\bra{#1}}
\def\del{\vec{\nabla}}
\newcommand{\avg}[1]{\langle#1\rangle}
\newcommand{\piecewise}[4]{\left\{\beginProtected{array}{rl}#1&:#2\\#3&:#4\endProtected{array}\right.}
\newcommand{\systeme}[2]{\left\{\beginProtected{array}{rl}#1\\#2\endProtected{array}\right.}
\def \KE{K\!E}
\def\Godel{G$\ddot{\mbox{o}}$del}

\onehalfspacing

\begin{document}

\answer{}

\textbf{101.1)} Show that a subgame perfect equilibrium of an extensive game $\Gamma$ is also a
subgame perfect equilibrium of the game obtained from $\Gamma$ by deleting a subgame not reached
in the equilibrium and assigning to the terminal history thus created the outcome of the equilibrium
in the deleted subgame.

By deleting that subgame, you can represent its value as the terminal history of the equilibrium
in that game, if that subgame is not reached by the $\Gamma$, then that subgame's equilibria is not
preferable to the one found for the entire game $\Gamma$, if it was then it would be the new equilibria,
since that is the definition of a subgame perfect equilibrium.  Recall that the subgame perfect
equilibrium takes its definition from:
\[
  O_h(s_{-i}^*|_h,s^*_i) \succsim_i|_h O_h(s_{-i}^*|_h,s_i)
\]
and that if there was a strategy that was preferable, it would be taken instead, and that would
be the new SPE.

\textbf{101.2)} Let $s$ be a strategy profile in an extensive game with perfect information $\Gamma$;
suppose that $P(h)=i$, $s_i(h)=a$, and $a'\in A(h)$ with $a'\neq a$. Consider the game $\Gamma'$
obtained from $\Gamma$ by deleting all histories of the form $(h,a',h')$ for some sequence of actions
$h'$ and let $s'$ be the strategy profile in $\Gamma'$ that is induced by $s$.  Show that if $s$ is
a subgame perfect equilibrium of $\Gamma$ then $s'$ is a subgame perfect equilibrium of $\Gamma'$.

\textbf{101.3)} Armies 1 and 2 are fighting over an island initially held by a battalion of army 2.
Army 1 has $K$ battalions and army 2 has $L$.  Whenever the island is occupied by one army the opposing
army can launch an attack.  The outcome of the attack is that the occupying battalion and one of the
attacking battalions are destroyed; the attacking army wins and, so long as it has battalions left,
occupies the island with one battalion.  The commander of each army is interested in maximizing the
number of surviving batallions but also regards the occupation of the island as worth more than one
battalion but less than two. (If, after an attack, neither army has any battalions left, then the
payoff of each commander is 0.) Analyse this situation as an extensive game and, using the notion of
subgame perfect equilibrium, predict the winner as a function of K and L.

We can see the first few moves of this game in the following tree:

\Tree [.1 [.2 [.1 [.2 ... (0,1) ] (1,0) ] (0,1) ] (1,0) ]

We will first examine the case where the armies are large, then we will examine the case where
they are not.  Suppose army 1 attacked army 2, Both armies would have one less batallion and
now army 1 would occupy the island.  Now the choice comes to player 2 whether or not to attack.
If he does he will lose a second battalion, which he doesnt want to do.  This means he abstains,
and player 1 wins the island.  So for $K\geq2$, the subgame perfect equilibrium
will be the strategy $s_1 = (attack, abstain, abstain, ...)$ and $s_2 = abstain$. If $K = 1,0$, 
then player 1 is going to abstain, because its better to not do anything than lose a battalion for
nothing, this looks like $s_1 = abstain$ and $s_2 = abstain$.

\textbf{282.1)} (Fighting an opponent of unknown strength) Two people are involved in a dispute.
Person 1 does not know whether person 2 is strong or weak; she assigns probability $\alpha$ to
person 2's being strong.  Person 2 is fully informed.  Each person can either fight or yield.
Each person's preferences are represented by the expected value of a Bernoulli payoff function
that assigns the payoff of 0 is she yields (regardless of the other person's action) and a payoff
of 1 if she fights and her opponent yields; if both people fight, then their payoffs are (-1,1) if
person 2 is strong and (1,-1) if person 2 is weak.  Formulate this situation as a Bayesian game
and find its Nash equilibria if $\alpha < \frac{1}{2}$ and if $\alpha > \frac{1}{2}$.


Incomplete information leads this problem to being represented as a Bayesian
Game.  We have two players, two actions for each players, and a collection of possible realities.
Either player 2 is strong, or player 2 is weak.  Either player can choose not to play, in which
case their payoff is 0, or they can fight, giving 1 if they win, or -1 if they lose.  player 1
will receive the same signal in each state, whereas player 2 will receive a different signal
depending on the state.  Player 1 will assign a belief $\alpha$ to whether or not player 2
is strong or weak.  The payoffs can be seen below:
\begin{center}
  \begin{tabular}{r |c|c|}
    \multicolumn{1}{r}{}
    & \multicolumn{1}{c}{Fight}
    & \multicolumn{1}{c}{Yield}\\
    \cline{2-3}
    Fight & -1,1 & 1,0\\
    \cline{2-3}
    Yield & 0,1 & 0,0\\
    \cline{2-3}
  \end{tabular}
  \begin{tabular}{r |c|c|}
    \multicolumn{1}{r}{}
    & \multicolumn{1}{c}{Fight}
    & \multicolumn{1}{c}{Yield}\\
    \cline{2-3}
    Fight & 1,-1 & 1,0\\
    \cline{2-3}
    Yield & 0,1 & 0,0\\
    \cline{2-3}
  \end{tabular}\\
  \vspace{2mm}
  \hspace{1cm}P2 Strong\hspace{2.4cm}P2 Weak

\end{center}


Player 1's payoff for yielding is 0, and for fighting is:
\[
  \alpha(-1) + (1-\alpha)(1) = 1 - 2\alpha
\]
Note that if $\alpha$ is less than $\frac{1}{2}$ then then payoff is greater than 0, and player 1
would prefer to fight, if $\alpha$ is greater than $\frac{1}{2}$ then the payoff will be less than
0, which is less than the payoff for not playing.  So if $\alpha<\frac{1}{2}$ then player one would
fight, and player 2 would choose the best response to player one fighting, which means player two
would yield if he was weak and fight if he was strong.  The NE in this case is therefore
\[
  \text{Player 1: Fight}
\]
\[
  \text{Player 2: (Fight, Yield)}
\]
If $\alpha>\frac{1}{2}$ than player 1 would yield, in which case, player two should always fight.
This can be represented as:
\[
  \text{Player 1: Yield}
\]
\[
  \text{Player 2: (Fight, Fight)}
\]


\textbf{282.2)} (An exchange game) Each of two individuals receives a ticket on which there is an
integer from 1 to $m$ indicating the size of a prize she may receive.  The individuals' tickets are
assigned randomly and independently; the probability of an individual's receiving each possible
number is positive.  Each individual is given the option of exchanging her prize for the other
individual's prize; the individuals are given this option simultaneously.  If both individuals wish
to exchange, then the prizes are exchanged; otherwise each individual receives her own prize.  Each
individual's objective is to maximize her expected monetary payoff.  Model this situation as a
Bayesian game and show that in any Nash equilibrium the highest prize that either individual is
willing to exchange is the smallest possible prize.

This is a Bayesian game with two players, two actions, and many states. The states are the collection
of possible ticket values $(a_1,a_2)$.  The actions are exchange or dont exchange.  The signal is
going to indicate each player's own prize size.  The belief function is going to be a function
mapping the signal $a_i$ to the state $(a_1,a_2)$.  This is the probability of that state given
the number on the player's own ticket.  Suppose player $i$ was going to exchange his ticket.
If he wants to exchange it means that player $j$ must know that player $i$ thinks that he has a
ticket value $a_i$ that is greater than $a_j$, so player $j$ has no reason to exchange.  The only
exception is when both players have $a = 1$, since no player has anything to lose in this situation,
and will always exchange.

\textbf{282.3)} (Adverse selection) Firm $A$ (the "acquirer") is considering taking over firm $T$ (the
"target").  It does not know firm $T$'s value; it believes that this value, when firm $T$ is controlled
by its own management, is at least \$0 and at most \$100, and assigns equal probability to each of the
101 dollar values in this range.  Firm $T$ will be worth 50\% more under firm $A$'s management than it
is under its own management.  Suppose that firm $A$ bids $y$ to take over firm $T$, and firm $T$ is
worth $x$ (under its own management).  Then if $T$ accepts $A$'s offer, $A$'s payoff is $\frac{3}{2}x-y$
and $T$'s payoff is $y$; if $T$ rejects $A$'s offer, $A$'s payoff is 0 and $T$'s payoff is $x$.  Model
this situation as a Bayesian game in which firm $A$ chooses how much to offer and firm $T$ decides the
lowest offer to accept.  Find the Nash equilibrium (equilibria?) of this game.  Explain why the logic
behind the equilibrium is called adverse selection.

We can model this game as a Bayesian game.  The two players are $A$ and $T$.  The actions for $A$ are
to make a bid $y$.  The actions for $T$ are deciding the possible values for which $T$ will accept the
bid $y$. The states are the values of $x$, the value of the firm $T$.  $A$ will receive the same signal
regardless of the state, $T$ will know the state that they are in.  The beliefs of $A$ are that each
value $x$ is equally distributed between 0 and 100.  To find the Nash equilibria of this game we will
first consider the case where $A$ bids a value $y$.  The values of $y$ for which firm $T$ will take the
deal are the values for which $x<y$.  If $x$ is larger than $y$, then they will get a higher payoff by
not taking the deal.

$A$ is going to make some assumption about the type of $T$.  If $A$ bids some value $y$, then all of the
possible values for the company $x$ will be between 0 and $y$.  Since they are distributed evenly between
0 and $y$ (only the bids that are below $y$ count, since they are the only ones that dont provide a payoff
of 0), $A$ expects $x$ to be of a value $\frac{y}{2}$.  From this we can calculate his expected payoff:
\[
  E(y) = \frac{3}{2}\frac{y}{2} - y = \frac{-y}{4}
\]
Which is less than 0.  So $A$'s best bet would be to offer 0.  And so the Nash equilibrium is to bid
nothing.  This is called adverse selection since if $A$ were to bid say, 50 dollars, on average he would
make 25 dollars.  Instead, $A$ selects something adverse to his payoff, because he tries to make
assumptions on the state of $T$.

\textbf{307.1)} (Swing voter's curse) Whether candidtae 1 or candidate 2 is elected depends on the votes
of two citizens.  The economy may be in one of two states, $A$ and $B$.  The citizens agree that candidate
1 is best if the state is $A$ and candidate 2 is best if the state is $B$.  Each citizen's preferences are
represented by the expected value of a Bernoulli payoff function that assigns a payoff of 1 if the best
candidate for the state wins (obtains more votes than the other candidate), a payoff of 0 if the other
candidate wins, and a payoff of $\frac{1}{2}$ if the candidates tie.  Citizen 1 is informed of the state,
wheres citizen 2 believes it is $A$ with probability 0.9 and $B$ with probability 0.1.  Each citizen may
either vote for candidate 1, vote for candidate 2, or not vote.

(a) Formulate this situation as a Bayesian game.

There are two players, 1 and 2.  Each player has 3 actions, vote for candidate 1, 2, or abstain.  The
possible states are the states of the economy, $A$ or $B$.  Player 1 gets a signal for each state and
assigns the probability of 1 to that state.  Player 2 gets no signal, and assigns a probability of 0.9
to $A$ and 0.1 to $B$.  The payoffs can be seen in the following matricies:

\begin{center}
  \begin{tabular}{r |c|c|c|}
    \multicolumn{1}{r}{}
    & \multicolumn{1}{c}{Can1}
    & \multicolumn{1}{c}{Can2}
    & \multicolumn{1}{c}{Abstain}\\
    \cline{2-4}
    Can1 & 1 & 1/2 & 1\\
    \cline{2-4}
    Can2 & 1/2 & 0 & 0\\
    \cline{2-4}
    Abstain & 1 & 0 & $\frac{1}{2}$\\
    \cline{2-4}
  \end{tabular}
    \begin{tabular}{r |c|c|c|}
    \multicolumn{1}{r}{}
    & \multicolumn{1}{c}{Can1}
    & \multicolumn{1}{c}{Can2}
    & \multicolumn{1}{c}{Abstain}\\
    \cline{2-4}
    Can1 & 0 & 1/2 & 0\\
    \cline{2-4}
    Can2 & 1/2 & 1 & 1\\
    \cline{2-4}
    Abstain & 0 & 1 & $\frac{1}{2}$\\
    \cline{2-4}
  \end{tabular}\\
  \vspace{2mm}
  \hspace{1.5cm}Economy A\hspace{4.3cm}Economy B

\end{center}

(b) Show that the game has exactly two pure Nash equilibria, in one of which citizen 2 does not
vote and in the other of which she votes for 1.

Citizen 1 will always vote for the candidate that best helps the economy.  For economy $A$ that is
candidate 1 and for economy $B$ that is candidate 2.  Note that this is regardless of the state of
the economy, it is always the better choice.  Citizen two is a different story, she believes
that we are in $A$ with probability 0.9.  Her best response in economy $A$ is to either vote for
candidate 1, or abstain.  In economy $B$, her best response is to vote for candidate 2, or abstain.
So clearly abstaining is a nash equilibria.  I dont think that her always voting for candidate 1 is
a Nash equilibrium, since she knows that in state $B$ she will be wrong.  Abstaining will always
provide her with a better outcome than not abstaining.  If she abstains she gets a payoff of 1, if
she votes for 1 she gets a payoff of $0.9\cdot1 + 0.1\cdot\frac{1}{2}$, which is clearly less than 1.

(c) Show that an action of one of the players in the second equilibrium is weakly dominated.

Again, I dont think this game has a second pure Nash equilibrium, but if it did, then the utility
of this equilibrium would be equal to the utility when playing the other equilibrium.

(d) Why is "swing voter's curse" an appropriate name for the determinant of citizen 2's decision in
the first equilibrium?

This is because player 2 knows that they do not have the full amount of information, and are
abstaining, leaving the entire decision to someone else.  it falls to player 1 to make the
decision, and it is a curse because only one person is making the decision, and if he makes
it wrong he will get all the blame.

\end{document}
