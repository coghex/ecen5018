\documentclass[12pt, notitlepage, final]{article} 

\newcommand{\name}{Vince Coghlan}

%\usepackage[dvips]{graphics,color}
\usepackage{amsfonts}
\usepackage{amssymb}
\usepackage{amsmath}
\usepackage{latexsym}
\usepackage{enumerate}
\usepackage{amsthm}
%\usepackage{nccmath}
\usepackage{setspace}
\usepackage[pdftex]{graphicx}
\usepackage{epstopdf}
%\usepackage[siunitx]{circuitikz}
\usepackage{tikz}
\usepackage{float}
%\usepackage{cancel} 
\usepackage{setspace}
%\usepackage{overpic}
\usepackage{mathtools}
\usepackage{listings}
\usepackage{color}
\usepackage{qtree}
%\usepackage{gensymb}

\usetikzlibrary{calc}
\usetikzlibrary{matrix}
\usetikzlibrary{positioning}

\numberwithin{equation}{section}
\DeclareRobustCommand{\beginProtected}[1]{\begin{#1}}
\DeclareRobustCommand{\endProtected}[1]{\end{#1}}
\newcommand{\dbr}[1]{d_{\mbox{#1BR}}}
\newtheorem{lemma}{Lemma}
\newtheorem*{corollary}{Corollary}
\newtheorem{theorem}{Theorem}
\newtheorem{proposition}{Proposition}
\theoremstyle{definition}
\newtheorem{define}{Definition}
\newcommand{\column}[2]{
\left( \begin{array}{ccc}
#1 \\
#2
\end{array} \right)}

\newdimen\digitwidth
\settowidth\digitwidth{0}
\def~{\hspace{\digitwidth}}

\setlength{\parskip}{1pc}
\setlength{\parindent}{0pt}
\setlength{\topmargin}{-3pc}
\setlength{\textheight}{9.0in}
\setlength{\oddsidemargin}{0pc}
\setlength{\evensidemargin}{0pc}
\setlength{\textwidth}{6.5in}
\newcommand{\answer}[1]{\newpage\noindent\framebox{\vbox{{\bf ECEN 5018 Spring 2014} 
\hfill {\bf \name} \vspace{-1cm}
\begin{center}{Homework \#3}\end{center} } }\bigskip }

\DeclareMathOperator*{\argmin}{arg\,min}

%absolute value code
\DeclarePairedDelimiter\abs{\lvert}{\rvert}%
\DeclarePairedDelimiter\norm{\lVert}{\rVert}
\makeatletter
\let\oldabs\abs
\def\abs{\@ifstar{\oldabs}{\oldabs*}}
%
\let\oldnorm\norm
\def\norm{\@ifstar{\oldnorm}{\oldnorm*}}
\makeatother

\def\dbar{{\mathchar'26\mkern-12mu d}}
\def \Frac{\displaystyle\frac}
\def \Sum{\displaystyle\sum}
\def \Int{\displaystyle\int}
\def \Prod{\displaystyle\prod}
%\def \P[x]{\Frac{\partial}{\partial x}}
%\def \D[x]{\Frac{d}{dx}}
\newcommand{\PD}[2]{\frac{\partial#1}{\partial#2}}
\newcommand{\PF}[1]{\frac{\partial}{\partial#1}}
\newcommand{\DD}[2]{\frac{d#1}{d#2}}
\newcommand{\DF}[1]{\frac{d}{d#1}}
\newcommand{\fix}[2]{\left(#1\right)_#2}
\newcommand{\ket}[1]{|#1\rangle}
\newcommand{\bra}[1]{\langle#1|}
\newcommand{\braket}[2]{\langle #1 | #2 \rangle}
\newcommand{\bopk}[3]{\langle #1 | #2 | #3 \rangle}
\newcommand{\Choose}[2]{\displaystyle {#1 \choose #2}}
\newcommand{\proj}[1]{\ket{#1}\bra{#1}}
\def\del{\vec{\nabla}}
\newcommand{\avg}[1]{\langle#1\rangle}
\newcommand{\piecewise}[4]{\left\{\beginProtected{array}{rl}#1&:#2\\#3&:#4\endProtected{array}\right.}
\newcommand{\systeme}[2]{\left\{\beginProtected{array}{rl}#1\\#2\endProtected{array}\right.}
\def \KE{K\!E}
\def\Godel{G$\ddot{\mbox{o}}$del}

\onehalfspacing

\begin{document}

\answer{}
I collaborated with Matthew Phillips on this assignment.

\textbf{101.1)} Show that a subgame perfect equilibrium of an extensive game $\Gamma$ is also a
subgame perfect equilibrium of the game obtained from $\Gamma$ by deleting a subgame not reached
in the equilibrium and assigning to the terminal history thus created the outcome of the equilibrium
in the deleted subgame.

To prove this we will begin recursively from the bottom of a extensive game.  At the last node of a
game, there will exist a move $a_i$ for a player $P(h) = i$ that exists in the subgame perfect
equilibrium set $s_i$.  This move will provide a payoff of some value and if it is not in the
subgame perfect equilibrium of $\Gamma$ it means that at the previous history that would take you
to the history $h$ can be replaced with the terminal history of $h$ given that move $a_i$.  In this
way the subgame perfect equilibrium has not changed, except for the fact that it is no longer defined
at the history $h$.  We can continue to delete nodes in this faction to attain a new tree.  If none
of these histories were in to overall subgame perfect equilibrium, then our new subgame perfect
equilibrium will be, by induction of the example we have shown, the exact same as before, with the
deleted histories removed from $s_i$.


\textbf{101.2)} Let $s$ be a strategy profile in an extensive game with perfect information $\Gamma$;
suppose that $P(h)=i$, $s_i(h)=a$, and $a'\in A(h)$ with $a'\neq a$. Consider the game $\Gamma'$
obtained from $\Gamma$ by deleting all histories of the form $(h,a',h')$ for some sequence of actions
$h'$ and let $s'$ be the strategy profile in $\Gamma'$ that is induced by $s$.  Show that if $s$ is
a subgame perfect equilibrium of $\Gamma$ then $s'$ is a subgame perfect equilibrium of $\Gamma'$.

Deleting all histories of the form $(h,a',h')$ will induce a strategy $s'$ that will contain actions
at each point not included past $a'$.  This since $s(h) = a$ and $a\neq a'$ we know that none of those
moves are going to be included in the new subgame perfect equilibrium.  Since the path $a'$ is not in
not in the subgame perfect equilibrium, then we know that the new subgame perfect equilibrium will
contain $s'(h)=a$.  In other words, there will be no difference at that point, since you would not
want to take $a'$ in the first place.  This being the case, all other individual histories not contained
in the subset $\Gamma'$ will be contained in the new SPE $s'$.

\textbf{101.3)} Armies 1 and 2 are fighting over an island initially held by a battalion of army 2.
Army 1 has $K$ battalions and army 2 has $L$.  Whenever the island is occupied by one army the opposing
army can launch an attack.  The outcome of the attack is that the occupying battalion and one of the
attacking battalions are destroyed; the attacking army wins and, so long as it has battalions left,
occupies the island with one battalion.  The commander of each army is interested in maximizing the
number of surviving batallions but also regards the occupation of the island as worth more than one
battalion but less than two. (If, after an attack, neither army has any battalions left, then the
payoff of each commander is 0.) Analyse this situation as an extensive game and, using the notion of
subgame perfect equilibrium, predict the winner as a function of K and L.

We can see the first few moves of this game in the following tree:

\Tree [.1 [.2 [.1 [.2 ... (0,1) ] (1,0) ] (0,1) ] (1,0) ]

We will first examine the case where the armies are large, then we will examine the case where
they are not.  Suppose army 1 attacked army 2, Both armies would have one less batallion and
now army 1 would occupy the island.  Now the choice comes to player 2 whether or not to attack.
If he does he will lose a second battalion, which he doesnt want to do.  This means he abstains,
and player 1 wins the island.  So for $K\geq2$, the subgame perfect equilibrium
will be the strategy $s_1 = (attack, abstain, abstain, ...)$ and $s_2 = abstain$. If $K = 1,0$, 
then player 1 is going to abstain, because its better to not do anything than lose a battalion for
nothing, this looks like $s_1 = abstain$ and $s_2 = abstain$.

\textbf{282.1)} (Fighting an opponent of unknown strength) Two people are involved in a dispute.
Person 1 does not know whether person 2 is strong or weak; she assigns probability $\alpha$ to
person 2's being strong.  Person 2 is fully informed.  Each person can either fight or yield.
Each person's preferences are represented by the expected value of a Bernoulli payoff function
that assigns the payoff of 0 is she yields (regardless of the other person's action) and a payoff
of 1 if she fights and her opponent yields; if both people fight, then their payoffs are (-1,1) if
person 2 is strong and (1,-1) if person 2 is weak.  Formulate this situation as a Bayesian game
and find its Nash equilibria if $\alpha < \frac{1}{2}$ and if $\alpha > \frac{1}{2}$.


Incomplete information leads this problem to being represented as a Bayesian
Game.  We have two players, two actions for each players, and a collection of possible realities.
Either player 2 is strong, or player 2 is weak.  Either player can choose not to play, in which
case their payoff is 0, or they can fight, giving 1 if they win, or -1 if they lose.  player 1
will receive the same signal in each state, whereas player 2 will receive a different signal
depending on the state.  Player 1 will assign a belief $\alpha$ to whether or not player 2
is strong or weak.  The payoffs can be seen below:
\begin{center}
  \begin{tabular}{r |c|c|}
    \multicolumn{1}{r}{}
    & \multicolumn{1}{c}{Fight}
    & \multicolumn{1}{c}{Yield}\\
    \cline{2-3}
    Fight & -1,1 & 1,0\\
    \cline{2-3}
    Yield & 0,1 & 0,0\\
    \cline{2-3}
  \end{tabular}
  \begin{tabular}{r |c|c|}
    \multicolumn{1}{r}{}
    & \multicolumn{1}{c}{Fight}
    & \multicolumn{1}{c}{Yield}\\
    \cline{2-3}
    Fight & 1,-1 & 1,0\\
    \cline{2-3}
    Yield & 0,1 & 0,0\\
    \cline{2-3}
  \end{tabular}\\
  \vspace{2mm}
  \hspace{1cm}P2 Strong\hspace{2.4cm}P2 Weak

\end{center}


Player 1's payoff for yielding is 0, and for fighting is:
\[
  \alpha(-1) + (1-\alpha)(1) = 1 - 2\alpha
\]
Note that if $\alpha$ is less than $\frac{1}{2}$ then then payoff is greater than 0, and player 1
would prefer to fight, if $\alpha$ is greater than $\frac{1}{2}$ then the payoff will be less than
0, which is less than the payoff for not playing.  So if $\alpha<\frac{1}{2}$ then player one would
fight, and player 2 would choose the best response to player one fighting, which means player two
would yield if he was weak and fight if he was strong.  The NE in this case is therefore
\[
  \text{Player 1: Fight}
\]
\[
  \text{Player 2: (Fight, Yield)}
\]
If $\alpha>\frac{1}{2}$ than player 1 would yield, in which case, player two should always fight.
This can be represented as:
\[
  \text{Player 1: Yield}
\]
\[
  \text{Player 2: (Fight, Fight)}
\]


\textbf{282.2)} (An exchange game) Each of two individuals receives a ticket on which there is an
integer from 1 to $m$ indicating the size of a prize she may receive.  The individuals' tickets are
assigned randomly and independently; the probability of an individual's receiving each possible
number is positive.  Each individual is given the option of exchanging her prize for the other
individual's prize; the individuals are given this option simultaneously.  If both individuals wish
to exchange, then the prizes are exchanged; otherwise each individual receives her own prize.  Each
individual's objective is to maximize her expected monetary payoff.  Model this situation as a
Bayesian game and show that in any Nash equilibrium the highest prize that either individual is
willing to exchange is the smallest possible prize.

This is a Bayesian game with two players, two actions, and many states. The states are the collection
of possible ticket values $(a_1,a_2)$.  The actions are exchange or dont exchange.  The signal is
going to indicate each player's own prize size.  The belief function is going to be a function
mapping the signal $a_i$ to the state $(a_1,a_2)$.  This is the probability of that state given
the number on the player's own ticket.  We can work this problem out for any state, The only
exception is when both players have $a = 1$, since no player has anything to lose in this situation,
they will always exchange.  We can split up each possibility that each player sees in the following
payoff matricies (NE are in \textbf{bold}):
\begin{center}
  \begin{tabular}{r |c|c|}
    \multicolumn{1}{r}{}
    & \multicolumn{1}{c}{Stay}
    & \multicolumn{1}{c}{Xchng}\\
    \cline{2-3}
    Stay & $\mathbf{a_1, a_2}$ & $\mathbf{a_1,a_2}$\\
    \cline{2-3}
    Xchng & $a_1$,$a_2$ & $a_2$,$a_1$\\
    \cline{2-3}
  \end{tabular}
    \begin{tabular}{r |c|c|}
    \multicolumn{1}{r}{}
    & \multicolumn{1}{c}{Stay}
    & \multicolumn{1}{c}{Xchng}\\
    \cline{2-3}
    Stay & $\mathbf{a_1,a_2}$ & $\mathbf{a_1,a_2}$\\
    \cline{2-3}
    Xchng & $\mathbf{a_1,a_2}$ & $\mathbf{a_2,a_1}$\\
    \cline{2-3}
  \end{tabular}
  \begin{tabular}{r |c|c|}
    \multicolumn{1}{r}{}
    & \multicolumn{1}{c}{Stay}
    & \multicolumn{1}{c}{Xchng}\\
    \cline{2-3}
    Stay & $\mathbf{a_1,a_2}$ & $\mathbf{a_1,a_2}$\\
    \cline{2-3}
    Xchng & $a_1$,$a_2$ & $a_2$,$a_1$\\
    \cline{2-3}
  \end{tabular}\\
  \vspace{2mm}
  \hspace{1.3cm}$a_1>a_2$\hspace{3.2cm}$a_1=a_2$\hspace{3.2cm}$a_1<a_2$
\end{center}

We can then analyse a player's payoffs (it works for both since they act symmetriclly).
\[
  E(\text{Exchanging}) = p_1\cdot {a_{2_1}} + p_2 \cdot {a_{1_2}} + p_3 \cdot a_{1_3}
\]
\[
  E(\text{Staying}) = p_{1}\cdot a_{1_1} + p_2 \cdot a_{1_2} + p_3 \cdot a_{1_3}
\]
These were determined by assuming the best response of the other player.  Most of these
values are the same, but $a_{2_1}<a_{1_1}$ which means that $E(\text{Exchanging})<E(\text{Staying})$,
and thus each player would rather take the safe route, and not exchange their tickets.  In other
words, each player is perfectly content switching their card if the other player has a higher value,
but when you have the higher value, it is simply not worth the risk.


\textbf{282.3)} (Adverse selection) Firm $A$ (the "acquirer") is considering taking over firm $T$ (the
"target").  It does not know firm $T$'s value; it believes that this value, when firm $T$ is controlled
by its own management, is at least \$0 and at most \$100, and assigns equal probability to each of the
101 dollar values in this range.  Firm $T$ will be worth 50\% more under firm $A$'s management than it
is under its own management.  Suppose that firm $A$ bids $y$ to take over firm $T$, and firm $T$ is
worth $x$ (under its own management).  Then if $T$ accepts $A$'s offer, $A$'s payoff is $\frac{3}{2}x-y$
and $T$'s payoff is $y$; if $T$ rejects $A$'s offer, $A$'s payoff is 0 and $T$'s payoff is $x$.  Model
this situation as a Bayesian game in which firm $A$ chooses how much to offer and firm $T$ decides the
lowest offer to accept.  Find the Nash equilibrium (equilibria?) of this game.  Explain why the logic
behind the equilibrium is called adverse selection.

This can be modeled as a Bayesian game.  The two players are $A$ and $T$.  The actions for $A$ are to
make a bid $y$.  The actions for $T$ are deciding the possible values for which $T$ will accept the bit
$Y$.  The actions for $T$ are deciding the possible values for which $T$ will accept the bid $y$.  The
states are the values of $x$, which are the values of the firm $T$.  $A$ will receive the same signal
regardless of the state, $T$ will know the state that they are in.  The beliefs of $A$ are that each
value $x$ is equally distributed between 0 and 100.  To find the Nash equilibria we will consider the
expected payoff of a bid $y$:
\[
  U_A(y) = \frac{1}{101}\sum_{x=0}^{y} \frac{3}{2}x - y
\]
Note that each payoff is going to include payoffs for every value of the company adjusted for the probability
of getting that value.  We can see a few of these values below:
\[
  U_A(0) = \frac{1}{101}\sum_{x=0}^{0} \frac{3}{2}x - 0 = 0
\]
A negative value regardless of the bid, for 1:
\[
  U_A(1) = \frac{1}{101}\sum_{x=0}^{1} \frac{3}{2}x - 1 = \frac{-1}{202}
\]
\[
  U_A(2) = \frac{1}{101}\sum_{x=0}^{2} \frac{3}{2}x - 2 = \frac{-3}{202}
\]
also negative.  in fact this sum can be represented as $\frac{-y(y+1)}{404}$, which is going to be less than
0 for any value.  The only exception is a bid larger then 100:
\[
  U_A(y) = \frac{3}{2}(50) - y = 75 - \text{number greater than 100}
\]
Which is clearly negative.  The equilibrium is when firm A bids 0 for the company since it gives a payoff
of zero, which is the only non negative payoff. This is called adverse selection since it would be common
sense to think that firm A could bid 50 and have a fifty percent chance of getting something and a fifty
percent chance of getting nothing.  Player A makes assumptions about the state of the world and then
selects a value that is adverse to its goals.

\textbf{307.1)} (Swing voter's curse) Whether candidtae 1 or candidate 2 is elected depends on the votes
of two citizens.  The economy may be in one of two states, $A$ and $B$.  The citizens agree that candidate
1 is best if the state is $A$ and candidate 2 is best if the state is $B$.  Each citizen's preferences are
represented by the expected value of a Bernoulli payoff function that assigns a payoff of 1 if the best
candidate for the state wins (obtains more votes than the other candidate), a payoff of 0 if the other
candidate wins, and a payoff of $\frac{1}{2}$ if the candidates tie.  Citizen 1 is informed of the state,
wheres citizen 2 believes it is $A$ with probability 0.9 and $B$ with probability 0.1.  Each citizen may
either vote for candidate 1, vote for candidate 2, or not vote.

(a) Formulate this situation as a Bayesian game.

There are two players, 1 and 2.  Each player has 3 actions, vote for candidate 1, 2, or abstain.  The
possible states are the states of the economy, $A$ or $B$.  Player 1 gets a signal for each state and
assigns the probability of 1 to that state.  Player 2 gets no signal, and assigns a probability of 0.9
to $A$ and 0.1 to $B$.  The payoffs can be seen in the following matricies:

\begin{center}
  \begin{tabular}{r |c|c|c|}
    \multicolumn{1}{r}{}
    & \multicolumn{1}{c}{Can1}
    & \multicolumn{1}{c}{Can2}
    & \multicolumn{1}{c}{Abstain}\\
    \cline{2-4}
    Can1 & 1 & 1/2 & 1\\
    \cline{2-4}
    Can2 & 1/2 & 0 & 0\\
    \cline{2-4}
    Abstain & 1 & 0 & $\frac{1}{2}$\\
    \cline{2-4}
  \end{tabular}
    \begin{tabular}{r |c|c|c|}
    \multicolumn{1}{r}{}
    & \multicolumn{1}{c}{Can1}
    & \multicolumn{1}{c}{Can2}
    & \multicolumn{1}{c}{Abstain}\\
    \cline{2-4}
    Can1 & 0 & 1/2 & 0\\
    \cline{2-4}
    Can2 & 1/2 & 1 & 1\\
    \cline{2-4}
    Abstain & 0 & 1 & $\frac{1}{2}$\\
    \cline{2-4}
  \end{tabular}\\
  \vspace{2mm}
  \hspace{1.5cm}Economy A\hspace{4.3cm}Economy B

\end{center}

(b) Show that the game has exactly two pure Nash equilibria, in one of which citizen 2 does not
vote and in the other of which she votes for 1.

Lets first look at the case where citizen 2 does not vote.  The best response in economy A would
be to vote for candidate A, since that would provide a payoff of 1.  Similarily the best response
in economy B would be to vote for candidate B, since that would also provide a payoff of 1.  This
means that the proposed NE would take the form $((1,2),\emptyset)$.  Let's see if citizen 1 had an
incentive to deviate.  If he votes for candidate 1 or 2 he will get a payoff of .95.  Not voting
provides an expected payoff of 1, which is best, so citizen 2 has no motivation to deviate.  Lets
check if the same phenomena occurs with citizen 1 when in economy A.  Voting for candidate 1 is
clearly the better choice, providing 1, instead of 1/2 or 0.  In economy B, 2 is the clear choice,
since the payoff would be 1, instead of 1/2 or 0.  In this way no player has any incentive to
deviate from this profile, and a Nash equilibrium emmerges.

Now lets look at the other equilibrium where player 2 always votes for candidate 1.  In this case
player 2's best response will ve to vote for candidate 2 when in economy B, and 1 or $\emptyset$
in economy A.  We can see that this proposes $((1,2),1)$ and $((\emptyset,2),1)$ as NE.  We can
dismiss the first one since in this case, player 2 would have an incentive to abstain from voting,
as that would increase the payoff from .95 to 1.  In the second case, deviating to voting for
candidate 2 would provide a payoff of .1 and abstaining would provide a payoff of .5.  We know
that citizen 2 has no incentive to deviate in this case.  In economy B citizen 1 clearly wants
to chose 2, since 1/2 is larger than the 0 payoff for deviating.  In economy A he is indifferent
to two different choices.  But he has no incentive to deviate, as he would be getting 1 by not
voting.  Since no player has any incentive to deviate, we know that $((\emptyset,2),1)$ is a NE.

The only thing left is to show that their isnt another NE.  The only other option would be for
citizen 1 to vote for candidate 2 all of the time.  In this case citizen 2's new best respose
would be to vote for candidate 1 while in economy A, and candidate 2 while in economy B.  He
could also not vote in economy B.  In all of these cases, player 2 would then have an incentive
to deviate to always voting for candidate 1, or abstaining, which would then put us in one of the
situations above.  In this way our two Nash equilibria are $((\emptyset,2),1)$ and
$((1,2),\emptyset)$.

(c) Show that an action of one of the players in the second equilibrium is weakly dominated.

In economy A player 1 will get a payoff of 1 if he abstains.  He will also get this payoff if he
switches to voting for candidate 1.  In this way player 1's strategy of abstaining is weakly
dominated by voting for candidate 1, since he is indifferent as to where he votes, but voting
for candidate 1 would give him better payoffs if player 2 was to alter his strategy.

(d) Why is "swing voter's curse" an appropriate name for the determinant of citizen 2's decision in
the first equilibrium?

This is because player 2 knows that they do not have the full amount of information, and are
abstaining, leaving the entire decision to someone else.  it falls to player 1 to make the
decision, and it is a curse because only one person is making the decision, and if he makes
it wrong he will get all the blame.

\end{document}
